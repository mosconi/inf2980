\documentclass{beamer}


\usepackage{url}
\usepackage{amsmath} 
\usepackage{amsthm} 
\usepackage{amssymb}
\usepackage[utf8]{inputenc}
\usepackage[brazil]{babel}
\usepackage{graphicx}
\usepackage{tikz}
\usepackage{tkz-graph}
\usetikzlibrary{matrix,arrows,decorations.pathmorphing}
\usepackage{pdflscape}
\usepackage{afterpage}
\usepackage{tabularx}

\author{Rodrigo Mosconi}
\title{Linear Ordering Problem}

\date{16 de Maio de 2016}

\makeindex
\begin{document}

\maketitle

\begin{frame}{Problema}
\end{frame}

\begin{frame}{Modelo}
\begin{itemize}
\item Matrizes estáticas
\item Uso de mapas 
\item Procurou-se usar conceitos de teoria de grupos
\end{itemize}
\end{frame}

\begin{frame}{Heurísticas utilizadas}
\framesubtitle{Solução inicial}
\begin{itemize}
\item Baseado no algortimo de Becker
\item O algoritmo não funciona se existirem linha nulas
\item Adicionado uma constante
\end{itemize}
\end{frame}

\begin{frame}{Heurísticas utilizadas}
\framesubtitle{Local Search}
\begin{itemize}
\item As vizinhanças são 2-ciclos (alguns)
\item Cálculo do objetivo pode ser eficiente
\item Opta pelo melhor vizinho
\end{itemize}
\end{frame}

\begin{frame}{Heurísticas utilizadas}
\framesubtitle{Iterated Local Search}
\begin{itemize}
\item Operação semelhante ao Local search
\item Diferencia a função objetivo e as vizinhanças
\item Vizinhanças são permutações não triviais
\item Opta pela melhor vizinhança
\item Pode interromper a procura prematuramente
\end{itemize}
\end{frame}

\begin{frame}{Implementação, experimentos e resultados}
\begin{itemize}
\item Implementado em Python
\item Dois modelos de ILS
\item Resultados não satisfatórios
\item Código e resultados em \url{https://github.com/mosconi/inf2980.git}
\end{itemize}
\end{frame}

\end{document}

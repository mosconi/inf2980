\documentclass[a4paper,10pt,onecolumn]{article}

\usepackage{url}
\usepackage{amsmath} 
\usepackage{amsthm} 
\usepackage{amssymb}
\usepackage[utf8]{inputenc}
\usepackage[brazil]{babel}
\usepackage{graphicx}
\usepackage{tikz}
\usepackage{tkz-graph}
\usetikzlibrary{matrix,arrows,decorations.pathmorphing}
\usepackage{pdflscape}
\usepackage{afterpage}
\usepackage{tabularx}
\usepackage[a4paper]{geometry}

\usepackage{listings}

\lstset{
  language=Python,
  basicstyle=\footnotesize,
  otherkeywords={self},
  captionpos=t,
  extendedchars=true, 
  frame=single,	  
  tabsize=4,	    
  title=\lstname,
  showstringspaces=false  
}


\newtheorem{defi}{Definição} 
\newtheorem{axioma}{Axioma}
\newtheorem{post}{Postulado} 
\newtheorem{prop}{Proposição}
\newtheorem{propri}{Propriedade}
\newtheorem{lem}{Lema} 
\newtheorem{theo}{Teorema}

\author{Rodrigo Mosconi}
\title{Linear Ordering Problem}
\date{16 de Maio de 2016}

\makeindex
\begin{document}

\maketitle
\tableofcontents

\section{Problema}

O problema de ordenação linear de um grafo direcionado $G=(V,E)$, sendo $V$ o
conjunto dos vértices e $E \subseteq V\times V$ o conjunto de arestas $(i,j)$
direcionadas de com custo $c_{ij}$, consiste em determinar uma permutação da
ordem dos vértices que maximize o custo das arestas diretas, ou não-reversas.

\section{Modelagem do problema}

O modelo para resolver este problema usa as seguintes premissas:
\begin{enumerate}
\item Toda aresta $(i,j)$ existente possui custo $c_{ij}>0$
\item As arestas não existentes possuem custo $0$
\item Não existe auto arestas, isto é, não existe arestas $(i,i)$
\end{enumerate}

Observando o problema sob a ótica da teoria de grupos, em especial, os grupos
de permutações $S_n$, onde $n$ é o número de vértices, é possível,
matematicamente, mapear a relação entre todos as permutações possíveis.
Entretanto, computacionalmente, pode não ser possível explorar todos as $n!$
permutações.  

Neste problema serão utilizadas heurísticas visando obter a melhor solução
possível e uma das premissas para as heurísticas é uma restrição na área de
procura.

A função objetivo associada ao  grafo $G=(V,E)$, em uma determinada ordem $\pi$:

\begin{equation}
  f(\pi)=\sum^n_{i=1}\sum^n_{j=i+1}c_{\pi_i\pi_j}
\end{equation}

Esta função é a soma do triangulo superior de uma matriz quadrada $n\times n$
dos custos associados das arestas.

Um grupo $(G,\cdot)$ é caracterizado como um conjunto $G$ de elementos e uma
operação $\cdot$ fechada entre estes elementos:

\begin{align*} G \times G & \rightarrow G \\ (a,b) & \mapsto a \cdot b
\end{align*}

  Este par de conjunto e operação deve satisfazer as propriedades:
\begin{enumerate}
\item A operação é associativa:\\
  $$ a \cdot ( b \cdot c) = ( a \cdot b) \cdot c \qquad \forall a,b,c \in G$$
\item Existe um e elemento neutro:\\
  $$ \exists e \in G, \textrm{ tal que }\, e \cdot a = a \cdot e = a \qquad \forall a \in G $$
\item Todo elemento possui um elemento inverso:\\
  $$ \forall a \in G, \exists b \in G,\textrm{ tal que }\, a \cdot b = b \cdot a = e $$
\end{enumerate}

Quando não houver ambiguidade, o grupo $(G,\cdot)$ será denotado apenas por
$G$ e a operação $a\cdot b$ será denotada por $ab$.

Das propriedades anteriores, pode-se deduzir que o elemento neutro é único e o
elemento inverso de $a \in G$ também é único.  O inverso de $a$ será denotado
por $a^{-1}$.

Um subconjunto $H$ de $G$ é um subgrupo de $(G,\cdot)$, denotado por $H < G$,
se também é grupo $(H,\cdot)$, sendo $\cdot$ a mesma operação.

Defini-se como \emph{ordem} de um grupo $G$ como o números de elementos em
$G$.  De forma análoga, a orderm de um subgrupo $H$ é o número de elementos de $H$.

Um grupo de permutações é definido como um grupo composto por um conjunto de
funções $Bij(C) = \{ f:C \rightarrow C| f \textrm{ é bijeção}\}$, sendo $C$
finito com $n$ elementos. A operação do grupo é a composição de funções
$\circ$.  Este grupo será denotado por $S_n$, e também é chamado de grupo
simétrico. A ordem deste grupo é $n!$.

Os elementos do conjunto do grupo $S_n$ podem ser representados na forma:

\begin{displaymath} \sigma= \bigl(
    \begin{smallmatrix} 1 & 2& 3& \cdots &n-1 &n \\ f(1) & f(2)& f(3)& \cdots
&f(n-1) &f(n) \\
    \end{smallmatrix} \bigr)
\end{displaymath}

Um elemento $\sigma$ é chamado de \emph{r-ciclo} de existem elementos
distintos $a_1,a_2,a_3,\dots,a_r \in \{1,2,3,\dots, n\}$ tais que:

\begin{enumerate}
\item{$\sigma(a_i) = a_{i+1}$ para $1 \leq i < r$;}
\item{$\sigma(a_r) = a_1$}
\item{$\sigma(b)=b$ para $b \in \{1,2,3,\dots, n\} \setminus
\{a_1,a_2,a_3,\dots,a_r\} $}
\end{enumerate}

Os $r-ciclos$ podem ser representados como $(a_1,a_2,a_3,\dots,a_r)$ e o
número $r$ é o comprimento do ciclo.  Os $2-ciclos$ também são chamados de
transposição.

Seja um $r-ciclo$ e um $s-ciclo$ quaisquer $\sigma-1,\sigma-2$ são ditos
disjuntos se para todos os elementos $a \in \{1,2,3,\dots, n\}$ tem-se que
$\sigma_1(a)=a $ ou $\sigma_2(a)=a$.  Em particular, para permutações
disjuntas, vale a propriedade comutativa, isto é: $\sigma_1 \circ \sigma_2 =
\sigma_2 \circ \sigma_1$.

Qualquer elemento de $S_n$ é resultado da composição de transposições.

\section{Heurísticas}

Para obter a melhor solução, usou-se de heurísticas para computar bons valores
e escolher o melhor deles.  Estes bons valores são denominados de ótimos
locais, isto é, em uma determinada região o ótimo local é a ordem que produz o
maior valor objetivo. De uma solução inicial, usou-se uma procura local
(\emph{local search} ou \emph{LS}) dentro de uma vizinhança bem definida para a obtenção do
ótimo local (\emph{local optima}). Após a obtenção de um ótimo local,
iterou-se entre as vizinhas, usando a heurística \emph{iterated local
  search} ou \emph{ILS}. A relação entre as vizinhanças também foi bem definido.

Para a obtenção de uma solução inicial, usou-se um algoritmo baseado no
algoritmo de 
Becker descrito em \cite{tommaso}.  Nas instâncias trabalhadas erra possível
que o denominador fosse $0$, tanto no numerador, quanto no denominador é
acrescido de uma unidade.

\begin{displaymath}
q_i = \dfrac{1+\sum_{k=1}^nc_{ik}}{1+\sum_{k=1}^nc_{ki}}
\end{displaymath}

A vizinhança para a procura local foi definida como todos as sequências que
estão a uma transposição de distância. E esta transposição é de somente
elementos consecutivos, ou seja, são 2-ciclos da forma $(i,i+1)$ para $1\leq
i<n$.  Foram implementados duas formas de cálculo do objetivo, uma primeira de
forma canônica e a outra seguindo o calculo da diferença\cite{tommaso}.

A vizinhança para o ILS, usou-se permutações que afetam $25\%$, $50\%$ ou
$100\%$, dos e elementos, de forma bem determinada. 

No local search, o critério de adoção de uma solução melhor foi usar o vizinho
que mais melhorava o objetivo.  Para o iterated local search, também usou a
adoção do melhor valor. Em um modelo, é realizado a exploração de todos as
vizinhanças e em outro, a procura é interrompida após passar por um número
pré-definido de vizinhanças sem melhorar o melhor valor.

\section{Experimentos}

A heurística proposta foi implementada em Python, pois é uma
linguagem que permitiu um rápido desenvolvimento.  Os testes foram executados
em um computador com procesasdor AMD A10 de 4GHz, 16GB de RAM e discos SSD
SATA 6Gbps em RAID 0.  O sistema operacional utilizado foi o Ubuntu 14.04 LTS
Server, com kernel Linux 4.2.0.  A versão do interpretador Python foi a 2.7.

Foram implementados 4 modelos, sendo o primeiro sempre usando o cálculo
completo do
objetivo na busca local e percorrendo todas as vizinhanças no ILS. Um segundo
modelo usa como critério a diferença de objetivo que a permutação resultará.
Um terceiro modelo obtém um primeiro ótimo local e em seguida computa em cada
grupo de permutações do ILS.  O quarto e último modelo possui com critério de
parada adicional se ao analizar $50$ vizinhanças consecutivas, o objetivo da
melhor não é melhorado.

Os modelos 2 e 4 propostos foram submetidos aos datasets IO\cite{ds_io},
RandA1\cite{ds_randa1} e Yagiura\cite{ds_yagiura}.  Cada modelo executou $10$
vezes cada datasets.

\section{Resultados}

Os resultados dos modelos 2 e 4 contra o dataset IO são mostrados na tabela
\ref{table:io}.
Os resultados dos modelos 2 e 4 contra o dataset RandA1 são mostrados na tabela
\ref{table:randa1}.
Os resultados dos modelos 2 e 4 contra o dataset Yagiura são mostrados na tabela
\ref{table:yagiura}.

\afterpage{
\clearpage
\newgeometry{left=2cm,right=2cm,top=2.5cm,bottom=2.5cm,footnotesep=0.5cm}
\thispagestyle{empty} 
\begin{table}[htbp]
  \noindent
  \centering
  \footnotesize
  \begin{tabular}{|c|c|ccccc|ccccc|}
    \hline 
    & & 
    \multicolumn{5}{c|}{Método 2} &
    \multicolumn{5}{c|}{Método 4} \\
    Nome da & Melhor valor & 
    && \multicolumn{3}{c|}{Tempo (s)} &
    && \multicolumn{3}{c|}{Tempo (s)} \\
    instância &  conhecido & 
    Resultado & gap(\%) & médio & máx & min &
    Resultado & gap(\%)  & médio & máx & min 
\\
    \hline
    N-be75eec & 236464 & 
    215641 & 8.81 & 2.06 & 2.38 & 1.95 &
    202257 & 14.47 & 0.72 & 0.57 & 0.79 
\\
    N-be75np & 716994 & 
696906 & 2.80 & 5.86   & 6.11  & 5.73 &
672690 & 6.18 & 2.18   & 2.30  & 1.86 
\\
    N-be75oi & 111171 & 
106286 & 4.39 & 1.02   & 1.04  & 1.01 &
106245 & 4.43 & 0.84   & 1.05  & 0.76 
\\
    N-be75tot & 980516 &
 941541 & 3.97 & 11.53   & 12.04  & 11.30  &
918895 & 6.28 & 6.07   & 6.95  & 5.51 
\\
    N-stabu70 & 362512 & 
327607 & 9.63 & 5.15   & 5.41  & 5.09 &
327607 & 9.63 & 7.14   & 7.33  & 7.05 
\\
    N-stabu74 & 541393 &
509895 & 5.82 & 9.61   & 9.73  & 9.52 &
476302 & 12.02 & 2.48   & 2.54  & 2.45 
\\
    N-stabu75 & 553303 &
511134 & 7.62 & 8.25   & 8.36  & 8.17 &
511134 & 7.62 & 11.30   & 11.45  & 11.04 
\\
    N-t59b11xx & 209320 &
195055 & 6.81 & 0.82   & 0.89  & 0.80 &
195055 & 6.81 & 0.83   & 0.85  & 0.82
\\
    N-t59d11xx & 147353 &
142310 & 3.42 & 3.47   & 3.89  & 3.29 &
142310 & 3.42 & 3.33   & 3.39  & 3.29
\\
    N-t59f11xx & 122520 &
118043 & 3.65 & 4.86   & 5.01  & 4.69 
&
117637 & 3.99 & 2.89   & 2.96  & 2.82 
\\
    N-t59i11xx & 8261545 &
8030372 & 2.80 & 2.46   & 2.61  & 2.39
&
8030372 & 2.80 & 2.52   & 2.63  & 2.49 
\\
    N-t59n11xx & 20928 &
20424 & 2.41 & 2.04   & 2.15  & 2.01&
19320 & 7.68 & 0.33   & 0.37  & 0.32 
\\
    N-t65b11xx & 356758 &
351676 & 1.42 & 15.36   & 15.90  & 15.10 &
350045 & 1.88 & 9.56   & 9.89  & 9.39
\\
    N-t65d11xx & 237739 &
228160 & 4.03 & 1.21   & 1.43  & 1.18 &
225217 & 5.27 & 0.57   & 0.57  & 0.56 
\\
    N-t65f11xx & 217295 &
212992 & 1.98 & 3.65   & 3.82  & 3.57 
&
212992 & 1.98 & 3.59   & 3.62  & 3.57 
\\
    N-t65i11xx & 14469163 &
14097638 & 2.57 & 4.12   & 4.73  & 3.99 
&
14097638 & 2.57 & 4.10   & 4.22  & 4.05 

\\
    N-t65l11xx & 16719 &
16410 & 1.85 & 0.51   & 0.51  & 0.51 
&
16410 & 1.85 & 0.52   & 0.53  & 0.51 
\\
    N-t65n11xx & 32157 &
31211 & 2.94 & 2.33   & 2.54  & 2.24 
&
30462 & 5.27 & 0.63   & 0.65  & 0.62 
\\
    N-t65w11xx & 138181029 &
132093649 & 4.41 & 4.82   & 4.95  & 4.75 &
132093649 & 4.41 & 4.88   & 5.03  & 4.79 
\\
    N-t69r11xx & 771149 &
736696 & 4.47 & 0.90   & 0.95  & 0.89  &
736696 & 4.47 & 0.91   & 0.93  & 0.90 
\\
    N-t70b11xx & 528419 &
512201 & 3.07 & 3.80   & 3.93  & 3.71 &
508910 & 3.69 & 1.96   & 2.03  & 1.95 
\\
    N-t70d11xx & 376725 &
365046 & 3.10 & 3.68   & 3.73  & 3.65 &
365046 & 3.10 & 3.77   & 3.97  & 3.71 
\\
    N-t70d11xxb & 366469 &
352145 & 3.91 & 6.52   & 6.63  & 6.44 &
352145 & 3.91 & 6.65   & 7.10  & 6.45 
\\
    N-t70f11xx & 360336 &
351712 & 2.39 & 6.26   & 6.45  & 6.18 &
343927 & 4.55 & 1.84   & 2.18  & 1.74 
\\
N-t70i11xx & 24785782 &
24061170 & 2.92 & 7.94   & 8.18  & 7.78 
&
23367918 & 5.72 & 0.75   & 0.82  & 0.73 
\\
N-t70k11xx & 60659200 &
59345600 & 2.17 & 1.74   & 1.80  & 1.71 
&
59345600 & 2.17 & 1.74   & 1.82  & 1.71 
\\
N-t70l11xx & 25253 &
25055 & 0.78 & 2.02   & 2.05  & 1.99 
&
25055 & 0.78 & 2.05   & 2.09  & 2.02 
\\
N-t70n11xx & 52704 &
51225 & 2.81 & 2.08   & 2.09  & 2.07 
&
50911 & 3.40 & 1.90   & 1.94  & 1.87 
\\
N-t70u11xx & 21716400 &
21576100 & 0.65 & 5.57   & 5.73  & 5.46 
&
20800600 & 4.22 & 1.84   & 1.87  & 1.82 
\\
N-t70w11xx & 224319954 &
221832872 & 1.11 & 15.26   & 15.56  & 14.93 
&
221832872 & 1.11 & 15.17   & 15.42  & 14.95 
\\
N-t70x11xx & 283808865 &
281117931 & 0.95 & 4.65   & 4.68  & 4.63 
&
276425213 & 2.60 & 3.81   & 3.86  & 3.78 
\\
N-t74d11xx & 566089 &
535664 & 5.37 & 6.62   & 6.67  & 6.57 
&
508209 & 10.22 & 0.91   & 0.92  & 0.90 
\\
N-t75d11xx & 578304 &
560084 & 3.15 & 8.34   & 9.32  & 6.62 
&
512713 & 11.34 & 1.04   & 1.05  & 1.02 
\\
N-t75e11xx & 2739219 &
2648431 & 3.31 & 3.16   & 3.59  & 2.63 
&
2648431 & 3.31 & 3.62   & 3.87  & 3.55 
\\
N-t75i11xx & 63567735 &
61316459 & 3.54 & 5.05   & 5.12  & 4.60 
&
61316459 & 3.54 & 5.15   & 5.31  & 5.06 
\\
N-t75k11xx & 108844 &
106706 & 1.96 & 1.54   & 2.38  & 1.13 
&
102237 & 6.07 & 0.39   & 0.40  & 0.39 
\\
N-t75n11xx & 93777 &
89960 & 4.07 & 2.44   & 2.68  & 1.94 
&
89120 & 4.97 & 1.09   & 1.10  & 1.08 
\\
N-t75u11xx & 52708323 &
51902071 & 1.53 & 8.60   & 8.98  & 8.30 
&
48667446 & 7.67 & 3.11   & 3.16  & 3.07 
\\
N-tiw56n54 & 91554 &
86579 & 5.43 & 15.26   & 15.46  & 15.03 
&
82322 & 10.08 & 2.44   & 2.50  & 2.40 
\\
N-tiw56n58 & 125224 &
116325 & 7.11 & 6.59   & 6.67  & 6.54 
&
114019 & 8.95 & 2.09   & 2.16  & 2.06 
\\
N-tiw56n62 & 176715 &
164941 & 6.66 & 4.78   & 4.87  & 4.75 
&
164941 & 6.66 & 4.79   & 4.94  & 4.73 
\\
N-tiw56n66 & 226547 &
213512 & 5.75 & 7.84   & 7.91  & 7.80 
&
211481 & 6.65 & 3.59   & 3.63  & 3.56 
\\
N-tiw56n67 & 226033 &
217719 & 3.68 & 12.94   & 13.15  & 12.74 
&
209188 & 7.45 & 3.70   & 3.75  & 3.65 
\\
    N-tiw56n72 & 365147 &
255144 & 30.13 & 11.73   & 11.79  & 11.64 &
248982 & 31.81 & 3.55   & 3.75  & 3.46 
\\
    N-tiw56r54 & 102948 &
93924 & 8.77 & 10.11   & 10.27  & 10.01 &
87459 & 15.05 & 1.24   & 1.25  & 1.23 
\\
    N-tiw56r58 & 129568 &
119368 & 7.87 & 8.19   & 8.33  & 8.10 &
116847 & 9.82 & 5.40   & 5.48  & 5.33
\\
    N-tiw56r66 & 209491 &
200990 & 4.06 & 8.21   & 8.38  & 8.14 &
190782 & 8.93 & 2.77   & 2.86  & 2.73 
\\
    N-tiw56r67 & 222810 &
215782 & 3.15 & 15.58   & 15.87  & 15.44 &
209253 & 6.08 & 1.89   & 1.91  & 1.88 
\\
    N-tiw56r72 & 270663 &
255144 & 5.73 & 11.73   & 11.79  & 11.6 &
248982 & 8.01 & 3.55   & 3.75  & 3.46 
\\
    N-usa79 & 1813986 &
1669702 & 7.95 & 22.91   & 25.08  & 22.15 &
1547768 & 14.68 & 8.57   & 10.87  & 7.92 
\\

\hline
  \end{tabular}
  \caption{Instâncias IO}
  \label{table:io}
\end{table}
\restoregeometry
\clearpage
}
\afterpage{
\clearpage
\newgeometry{left=2cm,right=2cm,top=2.5cm,bottom=2.5cm,footnotesep=0.5cm}
\thispagestyle{empty} 
\begin{table}[htbp]
  \noindent
  \footnotesize
  \centering
  \begin{tabular*}{\linewidth}{|c|c|ccccc|ccccc|}
    \hline 
    &  & 
    \multicolumn{5}{c|}{Método 2} &
    \multicolumn{5}{c|}{Método 4} \\
    Nome da & limites & 
    && \multicolumn{3}{c|}{Tempo (s)} &
    && \multicolumn{3}{c|}{Tempo (s)} \\
    &  conhecidos & 
    Resultado & gap(\%) & médio & máx & min &
    Resultado & gap(\%)  & médio & máx & min 
\\
    \hline

t1d100.01 & [106852,114468] &
96176 & 15.98 & 14.32   & 14.57  & 14.13 
&
95332 & 16.72 & 7.71   & 7.88  & 7.57 
\\
t1d100.02 & [105947,114077] &
94762 & 16.93 & 24.45   & 25.25  & 24.02 
&
92464 & 18.95 & 4.58   & 4.67  & 4.53 
\\
t1d100.03 & [109819,117843] &
98035 & 16.81 & 15.65   & 15.95  & 15.40 
&
95305 & 19.13 & 2.95   & 3.04  & 2.90 
\\
t1d100.04 & [109252,117639] &
99452 & 15.46 & 37.97   & 39.52  & 37.13 
&
96205 & 18.22 & 5.82   & 5.98  & 5.74 
\\
t1d100.05 & [108859,117538] &
98370 & 16.31 & 23.35   & 23.67  & 23.17 
&
95420 & 18.82 & 2.68   & 2.76  & 2.62 
\\
t1d100.06 & [108201,117057] &
98547 & 15.81 & 11.12   & 11.67  & 10.92 
&
96067 & 17.93 & 3.31   & 3.38  & 3.25 
\\
t1d100.07 & [108803,117118] &
98627 & 15.79 & 12.29   & 12.56  & 12.13 
&
94373 & 19.42 & 3.01   & 3.08  & 2.96 
\\
t1d100.08 & [107480,115756] &
97123 & 16.10 & 15.26   & 19.16  & 14.61 
&
95019 & 17.91 & 3.37   & 3.47  & 3.32 
\\
t1d100.09 & [108549,116527] &
96547 & 17.15 & 13.96   & 14.19  & 13.79 
&
93575 & 19.70 & 2.86   & 2.94  & 2.80 
\\
t1d100.10 & [108771,117518] &
98060 & 16.56 & 13.97   & 14.24  & 13.83 
&
96027 & 18.29 & 2.50   & 2.52  & 2.47 
\\
t1d100.11 & [107920,116637] &
98360 & 15.67 & 31.74   & 32.31  & 31.24 
&
95180 & 18.40 & 3.29   & 3.35  & 3.27 
\\
t1d100.12 & [108389,116617] &
98075 & 15.90 & 19.54   & 19.98  & 19.36 
&
93942 & 19.44 & 3.79   & 3.88  & 3.60 
\\
t1d100.13 & [108702,116816] &
99727 & 14.63 & 42.72   & 44.80  & 41.87 
&
94744 & 18.89 & 3.44   & 3.55  & 3.37 
\\
t1d100.14 & [105583,114199] &
93767 & 17.89 & 21.21   & 21.88  & 20.86 
&
91164 & 20.17 & 3.73   & 3.79  & 3.66 
\\
t1d100.15 & [108667,117221] &
98132 & 16.28 & 22.45   & 23.31  & 22.06 
&
95923 & 18.17 & 5.27   & 5.42  & 5.15 
\\
t1d100.16 & [107426,115781] &
96918 & 16.29 & 14.77   & 15.13  & 14.51 
&
95885 & 17.18 & 4.13   & 4.23  & 4.03 
\\
t1d100.17 & [105612,113860] &
96096 & 15.60 & 21.18   & 21.47  & 21.04 
&
92176 & 19.04 & 3.89   & 4.19  & 3.54 
\\
t1d100.18 & [107861,115959] &
96778 & 16.54 & 17.63   & 19.05  & 17.08 
&
94365 & 18.62 & 2.83   & 3.34  & 2.32 
\\
t1d100.19 & [108026,116987] &
99222 & 15.19 & 19.06   & 19.96  & 18.74 
&
96471 & 17.54 & 5.35   & 6.09  & 4.22 
\\
t1d100.20 & [109968,119008] &
99111 & 16.72 & 11.06   & 11.54  & 10.76 
&
96765 & 18.69 & 2.57   & 2.69  & 1.91 
\\
t1d100.21 & [107255,116326] &
96685 & 16.88 & 16.04   & 17.51  & 15.63 
&
92032 & 20.88 & 2.57   & 2.61  & 2.54 
\\
t1d100.22 & [108250,116987] &
97562 & 16.60 & 16.33   & 20.54  & 15.72 
&
95734 & 18.17 & 2.51   & 2.72  & 1.89 
\\
t1d100.23 & [106146,113264] &
94801 & 16.30 & 17.41   & 17.69  & 17.13 
&
93224 & 17.69 & 7.50   & 8.74  & 6.03 
\\
t1d100.24 & [108250,116959] &
98755 & 15.56 & 17.50   & 18.45  & 17.13 
&
95704 & 18.17 & 3.88   & 4.04  & 3.74 
\\
t1d100.25 & [106933,115311] &
97077 & 15.81 & 24.72   & 28.42  & 24.02 
&
95270 & 17.38 & 2.78   & 3.31  & 2.40 
\\


    \hline
  \end{tabular*}
  \caption{Instâncias RandA1}
  \label{table:randa1}
\end{table}
\restoregeometry
\clearpage
}

\afterpage{
\clearpage
\newgeometry{left=2cm,right=2cm,top=2.5cm,bottom=2.5cm,footnotesep=0.5cm}
\thispagestyle{empty} 
\begin{table}[htbp]
  \noindent
  \footnotesize
  \centering
  \begin{tabular*}{\linewidth}{|c|c|ccccc|ccccc|}
    \hline 
    &  & 
    \multicolumn{5}{c|}{Método 2} &
    \multicolumn{5}{c|}{Método 4} \\
    Nome da & limites & 
    && \multicolumn{3}{c|}{Tempo (s)} &
    && \multicolumn{3}{c|}{Tempo (s)} \\
    &  conhecidos & 
    Resultado & gap(\%) & médio & máx & min &
    Resultado & gap(\%)  & médio & máx & min 
\\
    \hline

n0500d001-1 & 58886 &
48601 & 17.47 & 284.34   & 301.54  & 234.29 
&
47472 & 19.38 & 72.07   & 79.59  & 57.96 
\\
n0500d001-2 & 52794 &
43967 & 16.72 & 317.59   & 383.26  & 259.14 
&
42778 & 18.97 & 76.70   & 82.70  & 74.91 
\\
n0500d001-3 & 54170 &
44809 & 17.28 & 405.04   & 442.99  & 328.40 
&
44028 & 18.72 & 74.25   & 77.41  & 58.00 
\\
n0500d001-4 & 59191 &
48851 & 17.47 & 311.53   & 355.03  & 266.62 
&
47651 & 19.50 & 75.86   & 78.02  & 74.18 
\\
n0500d001-5 & 58429 &
49593 & 15.12 & 354.16   & 418.09  & 304.30 
&
48203 & 17.50 & 65.34   & 76.40  & 52.59 
\\
n0500d005-1 & 234995 &
193167 & 17.80 & 434.97   & 453.48  & 329.17 
&
188877 & 19.63 & 68.40   & 75.57  & 55.16 
\\
n0500d005-2 & 234053 &
192690 & 17.67 & 454.41   & 457.19  & 451.56 
&
189386 & 19.08 & 70.73   & 75.98  & 58.88 
\\
n0500d005-3 & 237689 &
195631 & 17.69 & 457.70   & 464.88  & 451.91 
&
192247 & 19.12 & 71.37   & 77.52  & 60.91 
\\
n0500d005-4 & 238056 &
196530 & 17.44 & 377.56   & 381.38  & 375.47 
&
194476 & 18.31 & 64.50   & 75.17  & 54.80 
\\
n0500d005-5 & 234934 &
193493 & 17.64 & 425.77   & 463.63  & 418.73 
&
191387 & 18.54 & 87.73   & 99.49  & 69.02 
\\
n0500d010-1 & 434925 &
367453 & 15.51 & 394.78   & 396.98  & 393.65 
&
363638 & 16.39 & 78.10   & 81.06  & 66.03 
\\
n0500d010-2 & 433628 &
369909 & 14.69 & 796.61   & 877.61  & 763.72 
&
366948 & 15.38 & 88.43   & 101.40  & 69.72 
\\
n0500d010-3 & 433107 &
365688 & 15.57 & 406.26   & 423.81  & 397.39 
&
361746 & 16.48 & 70.47   & 80.49  & 56.26 
\\
n0500d010-4 & 435776 &
367987 & 15.56 & 479.35   & 485.01  & 472.03 
&
362580 & 16.80 & 71.15   & 80.22  & 63.79 
\\
n0500d010-5 & 434216 &
368760 & 15.07 & 480.64   & 486.00  & 477.44 
&
361135 & 16.83 & 65.25   & 80.10  & 55.41 
\\
n0500d050-1 & 2097339 &
1934858 & 7.75 & 961.14   & 1104.12  & 923.08 
&
1929434 & 8.01 & 122.89   & 145.13  & 100.54 
\\
n0500d050-2 & 2095533 &
1937223 & 7.55 & 1058.03   & 1085.49  & 1036.58 
&
1932745 & 7.77 & 326.59   & 353.37  & 285.03 
\\
n0500d050-3 & 2087570 &
\multicolumn{5}{|c|}{Sem resposta}
&
1912104 & 8.41 & 111.41   & 123.70  & 85.96 
\\
n0500d050-4 & 2091577 &
\multicolumn{5}{|c|}{Sem resposta}
&
1921721 & 8.12 & 113.54   & 126.50  & 89.12 
\\
n0500d050-5 & 2098933 &
\multicolumn{5}{|c|}{Sem resposta}
&
1928093 & 8.14 & 143.54   & 151.17  & 141.88 
\\
n0500d100-1 & 6514410 &
\multicolumn{5}{|c|}{Sem resposta}
&
6342167 & 2.64 & 335.16   & 347.47  & 329.65 
\\
n0500d100-2 & 6507318 &
\multicolumn{5}{|c|}{Sem resposta}
&
6342661 & 2.53 & 696.00   & 741.29  & 673.92 
\\
n0500d100-3 & 6524477 &
\multicolumn{5}{|c|}{Sem resposta}
&
6357582 & 2.56 & 880.94   & 939.07  & 857.81 
\\
n0500d100-4 & 6519873 &
\multicolumn{5}{|c|}{Sem resposta}
&
6352553 & 2.57 & 589.71   & 616.81  & 569.41 
\\
n0500d100-5 & 6514317 &
\multicolumn{5}{|c|}{Sem resposta}
&
6346307 & 2.58 & 396.00   & 432.37  & 366.68 
\\
    \hline
  \end{tabular*}
  \caption{Instâncias Yagiura}
  \label{table:yagiura}
\end{table}
\restoregeometry
\clearpage
}

\section{Conclusão}

A escolha das relações entre as vizinhanças em geral não foi satisfatória,
gerando muitos resultados afastados das melhores soluções já conhecidas.
Os códigos fontes, os datasets utilizados e resultados parciais estão
disponibilizados em \url{https://github.com/mosconi/inf2980.git}.

\begin{thebibliography}{99}
\bibitem{materialaula} Thibaut Vidal, \emph{Material de Aula}, 2016

\bibitem{sakuraba2010} Celso S. Sakuraba and Mutsunori Yagiura,
\emph{Efficient local search algorithms for the linear ordering problem},
International Transactions In Operational Research, 2010

\bibitem{tommaso} Tommaso Schiavinotto and Thomas Stützle, \emph{The Linear
Ordering Problem: Instances, Search Space Analysis and Algorithms}, Journal of
Mathematical Modelling and Algorithms, 2004

\bibitem{algebra} Arnaldo Garcia and Yves Lequain, \emph{Elementos de
Álgebra}, IMPA, 2005

\bibitem{ds_io}
\url{http://www.optsicom.es/lolib/lop/IO.zip}

\bibitem{ds_randa1}
\url{http://www.optsicom.es/lolib/lop/RandA1.zip}

\bibitem{ds_yagiura}
\url{http://www.al.cm.is.nagoya-u.ac.jp/~yagiura/lop/}



\end{thebibliography}

\end{document}


% HW: AMD A10 4.1 GHz 
% SO: linux ubuntu 14.04 LTS
% PL: python 2.7

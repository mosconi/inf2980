\documentclass[a4paper,10pt,twocolumn]{article}

\usepackage{url}
\usepackage{amsmath} 
\usepackage{amsthm} 
\usepackage{amssymb}
\usepackage[utf8]{inputenc}
\usepackage[brazil]{babel}
\usepackage{graphicx}
\usepackage{tikz}
\usepackage{tkz-graph}
\usetikzlibrary{matrix,arrows,decorations.pathmorphing}

\usepackage{listings}

\lstset{
  language=Python,
  basicstyle=\footnotesize,
  otherkeywords={self},
  captionpos=t,
  extendedchars=true, 
  frame=single,	  
  tabsize=4,	    
  title=\lstname,
  showstringspaces=false  
}


\newtheorem{defi}{Definição} 
\newtheorem{axioma}{Axioma}
\newtheorem{post}{Postulado} 
\newtheorem{prop}{Proposição}
\newtheorem{propri}{Propriedade}
\newtheorem{lem}{Lema} 
\newtheorem{theo}{Teorema}

\author{Rodrigo Mosconi}
\title{Linear order problem}
\date{16 de Maio de 2016}

\makeindex
\begin{document}

\maketitle
\tableofcontents

\section{Problema}
\section{Modelagem do problema}
\subsection{Teoria de grupos de permutações}
Um grupo é caracterizado como um conjunto $G$ de elementos e uma operação
fechada entre
estes elementos:

\begin{align*}
  G \times G & \rightarrow G \\
  (a,b) & \mapsto a \cdot b
\end{align*}

  Este par de conjunto e operação deve satisfazer as
propriedades:
\begin{enumerate}
\item A operação é associativa:\\
  $$ a \cdot ( b \cdot c) = ( a \cdot b) \cdot c \qquad \forall a,b,c \in G$$
\item Existe um e elemento neutro:\\
  $$ \exists e \in G, \textrm{ tal que }\, e \cdot a = a \cdot e = a \qquad \forall a \in G $$
\item Todo elemento possui um elemento inverso:\\
  $$ \forall a \in G, \exists b \in G,\textrm{ tal que }\, a \cdot b = b \cdot a = e $$
\end{enumerate}

Das propriedades anteriores, pode-se deduzir que o elemento neutro é único e o
elemento 
inverso de $a \in G$ também é único.  O inverso de $a$ será denotado por $a^{-1}$.

\section{Heurísticas}
\subsection{Local Search}
\subsection{Iterated Local Search}
\section{Experimentos}
\section{Resultados}
\section{Conclusão}

\begin{thebibliography}{99}
\bibitem{materialaula}
  Thibaut Vidal,
  \emph{Material de Aula},
  2016
\bibitem{sakuraba2010}
 Celso S. Sakuraba and Mutsunori Yagiura,
 \emph{Efficient local search algorithms for the linear
ordering problem},
 International Transactions In Operational Research,
 2010

\bibitem{tommaso}
  Tommaso Schiavinotto and Thomas Stützle,
  \emph{The Linear Ordering Problem: Instances, Search
Space Analysis and Algorithms},
  Journal of Mathematical Modelling and Algorithms,
  2004

\bibitem{algebra}
  Arnaldo Garcia and Yves Lequain,
  \emph{Elementos de Álgebra},
  IMPA,
  2005


\end{thebibliography}

\end{document}
